\documentclass[12pt]{article}

\usepackage[utf8]{inputenc}
\usepackage[russian]{babel}
\usepackage{geometry}
\usepackage{graphicx}
\usepackage{soul}
\usepackage{color}
\usepackage{colortbl}

\geometry{a4paper, top=15mm, bottom=15mm, left=10mm, right=10mm}


\begin{document}

\setcounter{page}{0}
\thispagestyle{empty}

\begin{center}
    Федеральное государственное автономное учебное учреждение высшего образования\\
    <<Национальный исследовательский университет ИТМО>>\\
\vspace{0.5cm}
    Мегафакультет компьютеных технологий и управления\\
    Факультет программной инженерии и компьютерной техники
\end{center}

\vspace{3cm}

\begin{center}
\Large
\textbf{
    Отчёт\\
    по лабораторной работе №4\\
    <<Исследование протоколов, форматов обмена информацией\\
    и языков разметки документов>>\\
    по дисциплине <<Информатика>>
}
\end{center}

\begin{center}
\large
    Вариант 13
\end{center}

\vspace{6cm}

\begin{flushright}
    Студент: Кожухин Иван Алексеевич, группа P3118\\
    Преподаватель: Рыбаков Степан Дмитриевич
\end{flushright}

\vspace{6cm}

\begin{center}
    Санкт-Петербург\\
    2022
\end{center}

\newpage

\tableofcontents

\newpage

\section{Задание}

1. Определить номер варианта как остаток деления на 36 порядкового номера в списке группы в ISU. В случае, если в данный день недели нет занятий, то увеличить номер варианта на восемь.\\
\\
2. Изучить форму Бэкуса--Наура.\\
\\
3. Изучить особенности языков разметки/форматов JSON, YAML, XML.\\
\\
4. Понять устройство страницы с расписанием для своей группы:\\
http://itmo.ru/ru/schedule/0/P3110/schedule.htm\\
\\
5. Исходя из структуры расписания конкретного дня, сформировать файл с расписанием в формате, указанном в задании в качестве исходного. При этом необходимо, чтобы в выбранном дне было не менее двух занятий (можно использовать своё персональное). В случае, если в данный день недели нет таких занятий, то увеличить номер варианта ещё на восемь.\\
\\
6. Обязательное задание (позволяет набрать до 65 процентов от максимального числа баллов БаРС за данную лабораторную): написать программу на языке Python 3.x, которая бы осуществляла парсинг и конвертацию исходного файла в новый. Нельзя использовать готовые библиотеки, в том числе регулярные выражения в Python и библиотеки для загрузки XML--файлов.\\
\\
7. Дополнительное задание №1 (позволяет набрать +10 процентов от максимального числа баллов БаРС за данную лабораторную).\\
\indent a) Найти готовые библиотеки, осуществляющие аналогичный парсинг и конвертацию файлов.\\
\indent b) Переписать исходный код, применив найденные библиотеки. Регулярные выражения также\\ нельзя использовать.\\
\indent c) Сравнить полученные результаты и объяснить их сходство/различие.\\
\\
8. Дополнительное задание №2 (позволяет набрать +10 процентов от максимального числа баллов БаРС за данную лабораторную).\\
\indent a) Переписать исходный код, добавив в него использование регулярных выражений.\\
\indent b) Сравнить полученные результаты и объяснить их сходство/различие.\\
\\
9. Дополнительное задание №3 (позволяет набрать +10 процентов от максимального числа баллов БаРС за данную лабораторную).\\
\indent a) Используя свою исходную программу из обязательного задания, программу из дополнительного задания №1 и программу из дополнительного задания №2, сравнить стократное время выполнения парсинга + конвертации в цикле.\\
\indent b) Проанализировать полученные результаты и объяснить их сходство/различие.\\
\\
10. Дополнительное задание №4 (позволяет набрать +5 процентов от максимального числа баллов БаРС за данную лабораторную).\\
\indent a) Переписать исходную программу, чтобы она осуществляла парсинг и конвертацию исходного файла в любой другой формат (кроме JSON, YAML, XML, HTML): PROTOBUF, TSV, CSV, WML и т. п.\\
\indent b) Проанализировать полученные результаты, объяснить особенности использования формата.\\

\newpage

\section{Ход работы}

Исходники кода программ и файлов:\\
https://github.com/troublegale/InfLab4\\

\begin{figure}[h]
    \centering
    \includegraphics{image1.png}
    \caption{QR-код для доступа к репозиторию на GitHub}
\end{figure}

\subsection{Обязательное задание}

Мой вариант предполагал создание программы, осуществляющей парсинг и конвертацию XML--файла в файл формата JSON.\\
\\
Содержимое исходного XML--файла:

\begin{verbatim}
    <?xml version="1.0" encoding="UTF-8"?>
    <schedule>
        <wednesday>
            <lesson>
                <subject>FPA</subject>
                <type>Lab</type>
                <professor>Tkeshelashvili N. M.</professor>
                <start_time>11:40</start_time>
                <end_time>13:10</end_time>
                <building>Kronverksky 49</building>
                <classroom>1330</classroom>
                <form>In-class + remote</form>
                <weeks>Odd</weeks>
            </lesson>
            <lesson>
                <subject>FPA</subject>
                <type>Lab</type>
                <professor>Tkeshelashvili N. M.</professor>
                <start_time>13:30</start_time>
                <end_time>15:00</end_time>
                <building>Kronverksky 49</building>
                <classroom>1330</classroom>
                <form>In-class + remote</form>
                <weeks>Odd</weeks>
            </lesson>
            <lesson>
                <subject>CS</subject>
                <type>Lab</type>
                <professor>Rybakov S. D.</professor>
                <start_time>11:40</start_time>
                <end_time>13:10</end_time>
                <building>Kronverksky 49</building>
                <classroom>2332</classroom>
                <form>In-class + remote</form>
                <weeks>Even</weeks>
            </lesson>
            <lesson>
                <subject>CS</subject>
                <type>Lab</type>
                <professor>Rybakov S. D.</professor>
                <start_time>13:30</start_time>
                <end_time>15:00</end_time>
                <building>Kronverksky 49</building>
                <classroom>2332</classroom>
                <form>In-class + remote</form>
                <weeks>Even</weeks>
            </lesson>
        </wednesday>
    </schedule>
\end{verbatim}

Для выполнения задания я создал несколько функций, что позволило мне с большим удобством работать с полученными при парсинге данными.\\

\begin{verbatim}
    def xml_to_array(name):
        file = open(name)
        file.readline()
        separator = 'Death Grips is online'
        separated = file.read().replace('\n', ' ').replace('\t', '')
        separated = separated.replace('> <', '>' + 
                                    separator + '<').replace('><', '>' + separator + '<')
        return separated.split(separator)


    def separate(data):
        separator = 'Death Grips is online'
        return data.replace('>', '<').replace('<', '>' + separator + 
                                                            '<').split(separator)[1:-1]
\end{verbatim}

\newpage

\begin{verbatim}
def get_tag(data):
    return separate(data)[0]


def just_tag(data):
    tag = separate(data)[0]
    return tag.replace('<', '').replace('>', '')


def get_value(data):
    if len(separate(data)) > 1:
        return separate(data)[1].replace('<', '').replace('>', '')
    return 0


def find_array_tags(arr):
    array_tags = []
    for i in arr:
        if i == get_tag(i) and arr.count(get_tag(i)) > 1 and just_tag(i) not in array_tags:
            array_tags.append(just_tag(i))
    return array_tags
\end{verbatim}

\newpage

Непосредственно процесс парсинга:

\begin{verbatim}
    json_file = open('timetable_str.json', 'w')
xml_array = xml_to_array('timetable.xml')
array_tags = find_array_tags(xml_array)

in_arr = False

for i in range(len(xml_array)):

    if i == 0:
        json_file.write('{')
    elif i == len(xml_array) - 1:
        json_file.write('}')
    else:

        data = xml_array[i]
        next_data = xml_array[i + 1]
        tag = get_tag(data)
        only_tag = just_tag(data)
        next_only_tag = just_tag(next_data)
        value = get_value(data)

        if only_tag in array_tags and not in_arr:
            json_file.write('"' + only_tag + '": [')
            in_arr = True

        if in_arr:
            if data == get_tag(data) and '/' not in data:
                json_file.write('{')
            if '/' in only_tag:
                if '/' + next_only_tag == only_tag:
                    json_file.write(', ')
                else:
                    json_file.write(']')
                    in_arr = False
            elif value:
                json_file.write('"' + only_tag + '": "' + value + ('", ' 
                                           if next_only_tag not in array_tags else '"}'))
        else:
            if '/' not in data:
                json_file.write('"' + only_tag + '": {')
            else:
                json_file.write('}')
\end{verbatim}


\newpage

Содержимое полученного файла в формате JSON (пропущенное через JSON-beautifier):

\begin{verbatim}
    {
        "wednesday": {
            "lesson": [
                {
                    "subject": "FPA",
                    "type": "Lab",
                    "professor": "Tkeshelashvili N. M.",
                    "start_time": "11:40",
                    "end_time": "13:10",
                    "building": "Kronverksky 49",
                    "classroom": "1330",
                    "form": "In-class + remote",
                    "weeks": "Odd"
                },
                {
                    "subject": "FPA",
                    "type": "Lab",
                    "professor": "Tkeshelashvili N. M.",
                    "start_time": "13:30",
                    "end_time": "15:00",
                    "building": "Kronverksky 49",
                    "classroom": "1330",
                    "form": "In-class + remote",
                    "weeks": "Odd"
                },
                {
                    "subject": "CS",
                    "type": "Lab",
                    "professor": "Rybakov S. D.",
                    "start_time": "11:40",
                    "end_time": "13:10",
                    "building": "Kronverksky 49",
                    "classroom": "2332",
                    "form": "In-class + remote",
                    "weeks": "Even"
                },
                {
                    "subject": "CS",
                    "type": "Lab",
                    "professor": "Rybakov S. D.",
                    "start_time": "13:30",
                    "end_time": "15:00",
                    "building": "Kronverksky 49",
                    "classroom": "2332",
                    "form": "In-class + remote",
                    "weeks": "Even"
                }
            ]
        }
    }
\end{verbatim}

\newpage

\subsection{Дополнительное задание №1}

Для выполнения этого задания я использовал библиотеки xmltodict и json. Код программы:

\begin{verbatim}
    xml_file = open('timetable.xml')
    json_file = open('timetable_add1.json', 'w')
    
    # parsing process
    timetable = xmltodict.parse(xml_file.read())
    
    # converting to json
    json_str = json.dumps(timetable)[::-1]
    json_str = json_str.replace('}', '', 1)
    json_str = json_str[::-1]
    root = json_str.split()[0] + ' '
    json_str = json_str.replace(root, '', 1)
    json_file.write(json_str)
\end{verbatim}

Содержимое результирующего файла полностью совпадает с содержимым файла, полученного по завершению работы программы для основного задания (форматом в одну строку).

\subsection{Дополнительное задание №2}

Для выполнения этого задания я скопировал код для основного задания и добавил в некоторых функциях работу с регулярными выражениями. Измененные фрагменты кода:

\begin{verbatim}
    def get_tag(data):
        tag = re.search(r'</*\w+>', data)[0]
        return tag
    
    
    def get_value(data):
        info = re.findall(r'[^<>\t]+', data)
        if len(info) == 3:
            return info[1]
        else:
            return 0
\end{verbatim}

Содержимое результирующего файла полностью совпадает с содержимым файла, полученного по завершению работы программы для основного задания (форматом в одну строку).

\newpage

\subsection{Дополнительное задание №3}

Для измерения времени работы программ я использовал библиотеку time. Я поместил части программ, в которых происходит парсинг и конвертация, в цикл, который выполнялся 100 раз. Перед этим циклом я присвоил переменной start\_time значение time.time(), зафиксировав время начало работы. После цикла я присвоил переменной end\_time значение time.time(), зафиксировав время завершения работы. Затем я вычислил разность end\_time и start\_time, округлив её до 3 знаков после запятой, и вывел это значение на экран. В результате замеров были получены следующие данные:\\

\begin{tabular}{|p{12cm}|p{5cm}|}
    \hline
    \centering
    \textbf{Метод выполнения задания} & \textbf{Среднее время работы программы}\\
    \hline
    Без использования сторонних библиотек и рег. выражений & $\approx 0.05$ сек\\
    \hline
    С использованием сторонних библиотек & $\approx 0.045$ сек\\
    \hline
    С использованием регулярных выражений & $\approx 0.063$ сек\\
    \hline
\end{tabular}
\\
\\
Вариант с использованием сторонних библиотек оказался самым быстродейственным, обогнав стандартный вариант на 0,005 секунды. Предположу, что это можно объяснить многократным использованием конструкции if/else в стандартном варианте, а также моей некомпетентностью в сфере оптимизации кода на языке Python. Вариант с использованием регулярных выражений оказался наименее быстродейственным, уступив стандартному целых 0,013 секунды. Регулярные выражения показали себя во всей красе.\\

\subsection{Дополнительное задание №4}

Для выполнения этого задания я выбрал формат TSV (tab-separated values). Большинство функций я скопировал из кода программы для основного задания, а вот фрагмент кода с конвертацией файла:

\begin{verbatim}
def xml_to_array(name):
    file = open(name)
    file.readline()
    separator = 'Death Grips is online'
    separated = file.read().replace('\n', ' ').replace('\t', '')
    separated = separated.replace('> <', '>' + separator + '<').replace('><', '>' + separator + '<')
    return separated.split(separator)


def separate(data):
    separator = 'Death Grips is online'
    return data.replace('>', '<').replace('<', '>' + separator + '<').split(separator)[1:-1]


def get_tag(data):
    return separate(data)[0]


def just_tag(data):
    tag = separate(data)[0]
    return tag.replace('<', '').replace('>', '')
\end{verbatim}
\newpage
\begin{verbatim}

def get_value(data):
    if len(separate(data)) > 1:
        return separate(data)[1].replace('<', '').replace('>', '')
    return 0


# parsing process
tsv_file = open('timetable.tsv', 'w')
xml_array = xml_to_array('timetable.xml')

# converting to tsv
for i in range(len(xml_array) - 1):
    if get_value(xml_array[i]):
        tsv_file.write(just_tag(xml_array[i]) + ('\t' if get_value(xml_array[i + 1]) else '\n'))
        if not get_value(xml_array[i + 1]):
            break
for i in range(len(xml_array) - 1):
    if get_value(xml_array[i]):
        tsv_file.write(get_value(xml_array[i]) + ('\t' if get_value(xml_array[i + 1]) else '\n'))

# result representation
tsv_file.close()
result = open('timetable.tsv')
print(result.read())
\end{verbatim}

Содержимое полученного TSV--файла (элементы каждой строки разделены знаком табуляции):

\begin{verbatim}
    subject	type	professor	start_time	end_time	building	classroom	form	weeks
    FPA	Lab	Tkeshelashvili N. M.	11:40	13:10	Kronverksky 49	1330	In-class + remote	Odd
    FPA	Lab	Tkeshelashvili N. M.	13:30	15:00	Kronverksky 49	1330	In-class + remote	Odd
    CS	Lab	Rybakov S. D.	11:40	13:10	Kronverksky 49	2332	In-class + remote	Even
    CS	Lab	Rybakov S. D.	13:30	15:00	Kronverksky 49	2332	In-class + remote	Even
\end{verbatim}

Конвертация в TSV--формат оказалась проще, чем в формат JSON, во многом потому, что в формате TSV данные разделены только знаками табуляции и по строчкам.

\newpage

\section{Вывод}

В ходе лабораторной работы я изучил форму Бэкуса--Наура, изучил особенности языков разметки XML, JSON и TSV, научился парсингу данных с использованием языка Python, узнал о различиях работы без использования дополнительных библиотек и с ним.

\section{Список использованной литературы}

1. Балакшин П. В. Информатика: лабораторные работы и тесты / П.В. Балакшин, В.В. Соснин, И.В. Калинин, Т.А. Малышева, С.В. Раков, Н.Г. Рущенко, А.М. Дергачев – СПб.: Университет ИТМО, 2019. - 59 с.\\
2. Сервис JSONOnlineTools [электронный ресурс]. – Режим доступа: https://onlinejsontools.com/ (дата обращения: 27.11.2022)\\
3. Сервис Hilite.me [электронный ресурс]. – Режим доступа: http://hilite.me/ (дата образения:\\
27.11.2022)\\
4. Лямин А.В., Череповская Е.Н. Объектно-ориентированное программирование. Компьютерный практикум. – СПб.: Университет ИТМО, 2017. - 143 с. – Режим доступа:\\ https://books.ifmo.ru/file/pdf/2256.pdf (дата обращения: 27.11.2022)

\end{document}